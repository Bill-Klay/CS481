\documentclass[12pt]{article}
\usepackage{fullpage}
\usepackage{tabularx}
\pagenumbering{gobble}

\begin{document}

\begin{center}

	\uppercase{\section*{CS481 Data Science}}
	\Large{Lab 5 Decision Trees} \\
	\large{\today} \\
	\rule{400pt}{1pt}
	\Large{Muhammad Bilal Khan \hfill 16K3778}

\end{center}

\section*{Decision Trees Results}
\subsection*{10 Fold Corss Validation}
\vspace{0.5cm}

\begin{center}
\begin{tabular}{ | m{8em} | m{2cm}| m{2cm} | m{2cm} | m{2cm} | m{2cm} |} 
\hline
 & Dataset 1 & Dataset 2 & Dataset 3 & Dataset 4 & Dataset 5 \\ 
\hline
DT using gini (without pruning) & 93.93  & 89.1  & 85.71  & 89.28  & 75.00 \\ 
\hline
DT using gini (with pruning) & 95.38  & 31.75  & 100.00  & 82.14  & 84.21  \\ 
\hline
DT using entropy (without pruning) & 93.84  & 90.05  & 90.47  & 89.28  & 78.94  \\
\hline
DT using entropy (with pruning) & 95.38  & 61.45  & 100.00  & 85.71  & 73.68  \\
\hline
Standard Deviation & 0.863  & 27.61 & 7.14  & 3.418  & 4.729  \\
\hline
\end{tabular}
\end{center}

\subsection*{70/30 Hold Out Approach}
\vspace{0.5cm}

\begin{center}
\begin{tabular}{ | m{8em} | m{2cm}| m{2cm} | m{2cm} | m{2cm} | m{2cm} |} 
\hline
 & Dataset 1 & Dataset 2 & Dataset 3 & Dataset 4 & Dataset 5 \\ 
\hline
DT using gini (without pruning) & 87.24  & 87.31  & 84.12  & 78.57  & 59.32  \\ 
\hline
DT using gini (with pruning) & 91.32  & 32.03  & 80.95  & 77.38  & 67.79  \\ 
\hline
DT using entropy (without pruning) & 87.24  & 87.86  & 85.71  & 77.38  & 64.40  \\
\hline
DT using entropy (with pruning) & 90.30  & 60.85  & 85.71  & 73.80  & 69.49  \\
\hline
Standard Deviation & 2.102 & 26.510 & 2.243 & 2.065 & 4.483 \\
\hline
\end{tabular}
\end{center}

\subsection*{K Fold Cross Validation vs. Hold Out Approach}

Regarding the comparision between CV and hold out, \emph{Cross Validation proves to be superior for smaller datasets} due to its results providing summarized results with comparision to all the dataset being divided in to training and testing set eventually, but due to comparision of K blocks with the rest of the data complexity becomes far more wide spread. \\
\emph{Hold out approach is considered better for bigger datasets}, since it constantly divides the data in to a considerable bigger chunk of usually 70/30 for training and testing respectively.

\end{document}
